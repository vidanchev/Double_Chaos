\documentclass[a4paper]{article}

\usepackage[margin=1in]{geometry} % full-width

\usepackage{amsmath}
\usepackage{amsthm}
\usepackage{amssymb}
\usepackage[utf8]{inputenc}
\usepackage{hyperref}
\usepackage{graphicx, color}
\usepackage[caption=false]{subfig}

% Author info
\title{Double Pendulum equations of motion and numerical results}
\author{Victor I.  Danchev \\ \href{vidanchev@uni-sofia.bg}{vidanchev@uni-sofia.bg}}

\date{
	Sofia University St.  Kliment Ohridski \\ 
	\today
}

\begin{document}
	\maketitle
	
	\begin{abstract}
	To be abstracted :)
	\end{abstract}

	\tableofcontents

	\newpage
	
	\section{Introduction} \label{Intro}
	To be introduced :)

	I'll be modelling a pendulum composed of two rigid rods as opposed to point masses hanged on ropes.

	\section{Equations of motion}\label{EoM}
	\textbf{Starting with the equations as derived on paper, will fill with explanation after!}
	
	\subsection{The Lagrangian}\label{Lagrangian_subsection}
	We'll define each rod $ i = \{ 1 , 2 \} $ as having length $(l_i)$, mass $(m_i)$ and moment of inertia about its center of mass ($I_{c i}$).
	Additionally we will have the gravitational acceleration norm $g$ as a constant ($g \cong 9.8 [\mathrm{m/s^2}]$).
	Normal vectors for the two "arms" of the pendulum are 
	\begin{eqnarray}\label{n_1_def}
		\hat{n}_1 &=& ( \sin{ \theta } , -\cos{ \theta } )^T \\ \label{n_2_def}
		\hat{n}_2 &=& ( \sin{ \varphi } , - \cos{ \varphi } )^T.
	\end{eqnarray}
	Positions of the centre of mass for the two rods are
	\begin{eqnarray}\label{rc_1_def}
		\vec{r}_{c1} &=& \frac{ l_1 }{ 2 }\hat{n}_1 \\ \label{rc_2_def}
		\vec{r}_{c2} &=& l_1 \hat{n}_1 + \frac{ l_2 }{ 2 }\hat{n}_2.
	\end{eqnarray}
	The Lagrangian is taken as the sum of kinetic terms minus the sum of potential terms, which gives
	\begin{eqnarray}\label{initial_Lagrangian}
		L = \frac{1}{2}m_1 |\dot{\vec{r}}_{c1}|^2 + \frac{1}{2}m_2 |\dot{\vec{r}}_{c2}|^2 + \frac{1}{2}I_{c1} |\vec{\omega}_{c1}|^2 + \frac{1}{2}I_{c2} |\vec{\omega}_{c2}|^2 - m_1 g y_{c1} - m_2 g y_{c2}, 
	\end{eqnarray}
	where $\dot{\vec{\omega}}_{ci}$ are the angular rates around the centres of mass for the two rods.
	Note that constant terms have been neglected (since a Lagrandian is unique up to linear transformation with constants).
	It is easy to see that $|\vec{\omega}_{c1}| = \dot{\theta}$ and $|\vec{\omega}_{c2}| = \dot{\varphi}$.
	Also, given a homogeneous rod of mass $m$ and length $l$, the moment of inertia around its centre of mass is 
	\begin{eqnarray}
		I_c = \frac{1}{12}m l^2.
	\end{eqnarray}
	Using \eqref{n_1_def}--\eqref{rc_2_def} in \eqref{initial_Lagrangian}, one gets the full form
	\begin{eqnarray}\label{full_form_Lagrangian}
		L =& \left( \frac{1}{6}m_1 l_1^2 + \frac{1}{2}m_2 l_2^2 \right) \dot{\theta}^2 + \frac{1}{6}m_2 l_2^2 \dot{\varphi}^2 + \frac{1}{2}m_2 l_1 l_2 \cos( \theta - \varphi ) \dot{\theta} \dot{\varphi} \\ \nonumber
		& + \frac{1}{2}m_1 g l_1 \cos{\theta} + m_2 g \left( l_1 \cos{\theta} + \frac{l_2}{2} \cos{\varphi} \right).
	\end{eqnarray}
	A lot of the clutter comes from constants, so I've chosen to simplify to
	\begin{eqnarray}\label{simplified_Lagrangian}
		L = a_{\theta}\dot{\theta}^2 + a_{\varphi}\dot{\varphi}^2 + a_{\mathrm{mix}}\cos( \theta - \varphi )\dot{\theta}\dot{\varphi} + b_{\theta}\cos{\theta} + b_{\varphi}\cos{\varphi},		
	\end{eqnarray}
	where I've defined the constants
	\begin{eqnarray}\label{a_th_def}
		a_{\theta} &=& \frac{1}{6}m_1 l_1^2 + \frac{1}{2}m_2 l_2^2 \\ \label{a_phi_def}
		a_{\varphi} &=& \frac{1}{6}m_2 l_2^2 \\ \label{a_mix_def}
		a_{\mathrm{mix}} &=& \frac{1}{2}m_2 l_1 l_2 \\ \label{b_th_def}
		b_{\theta} &=& l_1 g \left( \frac{m_1}{2} + m_2 \right) \\ \label{b_phi_def}
		b_{\varphi} &=& \frac{1}{2} l_2 g m_2.
	\end{eqnarray}
	At this point the equations of motion can be readily derived.
	
	\subsection{The Equations of Motion}\label{EoM_subsection}
	Lagrange's equations from a minimum action principle give dynamical equations
	\begin{eqnarray}\label{Lagrange_Equations}
		\frac{d}{dt}\left(\frac{d L}{d \dot{q}^i} \right) = \frac{dL}{dq^i},
	\end{eqnarray}
	where $q^i$ are the general positions (configuration space parameters) and $\dot{q}^i$ are their derivatives (general velocities).
	The combination of all of these $\{ q^1, q^2, ... , q^N, \dot{q}^1, \dot{q}^2, ... , \dot{q}^N \}$ is the full state of the system (phase space).
	In our case $N = 2$ and the state is $\vec{s} = ( \theta , \varphi , \dot{\theta} , \dot{\varphi} )^T$.
	I will further denote $\dot{\theta} \equiv \omega_{\theta}$ and $\dot{\varphi} \equiv \omega_{\varphi}$.
	
	Computing the equations \eqref{Lagrange_Equations} for the Lagrangian \eqref{simplified_Lagrangian} yields the equations of motion
	\begin{eqnarray}\label{eom_1}
		2 a_{\theta} \dot{\omega}_{\theta} + a_{\mathrm{mix}}\cos( \varphi - \theta ) \dot{\omega}_{\varphi} & = - b_{\theta}\sin{\theta} + a_{\mathrm{mix}}\sin( \varphi - \theta )\omega^2_{\varphi} = f_1( \mathrm{state} ) \\ \label{eom_2}
		2 a_{\varphi} \dot{\omega}_{\varphi} + a_{\mathrm{mix}}\cos( \varphi - \theta ) \dot{\omega}_{\theta} & = - b_{\varphi}\sin{\varphi} - a_{\mathrm{mix}}\sin( \varphi - \theta )\omega^2_{\theta} = f_2( \mathrm{state} ),
	\end{eqnarray}
	where $\mathrm{state} \equiv ( \theta , \varphi , \omega_{\theta} , \omega_{\varphi} )^T$.
	This can be written as a matrix equation
	\begin{eqnarray}
		\begin{pmatrix}
			2 a_{\theta} & a_{\mathrm{mix}}\cos( \varphi - \theta ) \\
			a_{\mathrm{mix}}\cos( \varphi - \theta ) & 2 a_{\varphi} 
		\end{pmatrix} 
		\begin{pmatrix}
			\dot{\omega}_{\theta} \\
			\dot{\omega}_{\varphi}
		\end{pmatrix} 
		& = & 
		\begin{pmatrix}
			f_1( \mathrm{state} ) \\
			f_2( \mathrm{state} ).
		\end{pmatrix}
	\end{eqnarray}
	Complementing with the two equations $\dot{\varphi} = \omega_{\varphi}$ and $\dot{\theta} = \omega_{\theta}$ gives us a complete 1st order system of 4 non-linear differential equations.
	To solve it explicitly, the matrix on the left-hand-side should be invertable.
	\begin{eqnarray}
		A_{\mathrm{LHS}} & = &
		\begin{pmatrix}
			2 a_{\theta} & a_{\mathrm{mix}}\cos( \varphi - \theta ) \\
			a_{\mathrm{mix}}\cos( \varphi - \theta ) & 2 a_{\varphi} 
		\end{pmatrix} 
	\end{eqnarray}
	Looking at the determinant, the matrix will always be invertable as long as
	\begin{eqnarray}
		\det{ A_{\mathrm{LHS}} } = 4 a_{\theta} a_{\varphi} - a^2_{\mathrm{mix}} \cos^2( \varphi - \theta ) \neq 0.
	\end{eqnarray}
	Given the definitions of the constants $a$ and $b$, this statement is equivalent to
	\begin{eqnarray}
		\cos^2( \varphi - \theta ) \neq \frac{ 4a_{\theta} a_{\varphi} }{ a^2_{\mathrm{mix}} } = \frac{4}{3} \left( \frac{ l_2 }{ l_1 } \right)^2 \left[ 1 + \frac{m_1 l_1^2}{3 m_2 l_2^2} \right],
	\end{eqnarray}
	which we'll show holds for all the physical cases.

	To be continued :)
	
	\section{Conclussions}
	To be concluded :)

	\newpage
	\begin{thebibliography}{9}

	\bibitem{Goldstein}
	H.~Goldstein, C.~Poole and J.~Safko, \textit{Classical Mechanics}.

	\end{thebibliography}

\end{document}